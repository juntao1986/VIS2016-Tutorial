%\documentclass[journal]{vgtc}                % final (journal style)
%\documentclass[review,journal]{vgtc}         % review (journal style)
%\documentclass[widereview]{vgtc}             % wide-spaced review
\documentclass[preprint,journal]{vgtc}       % preprint (journal style)
%\documentclass[electronic,journal]{vgtc}     % electronic version, journal

%% Uncomment one of the lines above depending on where your paper is
%% in the conference process. ``review'' and ``widereview'' are for review
%% submission, ``preprint'' is for pre-publication, and the final version
%% doesn't use a specific qualifier. Further, ``electronic'' includes
%% hyperreferences for more convenient online viewing.

%% Please use one of the ``review'' options in combination with the
%% assigned online id (see below) ONLY if your paper uses a double blind
%% review process. Some conferences, like IEEE Vis and InfoVis, have NOT
%% in the past.

%% Please note that the use of figures other than the optional teaser is not permitted on the first page
%% of the journal version.  Figures should begin on the second page and be
%% in CMYK or Grey scale format, otherwise, colour shifting may occur
%% during the printing process.  Papers submitted with figures other than the optional teaser on the
%% first page will be refused.

%% These three lines bring in essential packages: ``mathptmx'' for Type 1
%% typefaces, ``graphicx'' for inclusion of EPS figures. and ``times''
%% for proper handling of the times font family.

\usepackage{mathptmx}
\usepackage{graphicx}
\usepackage{times}
\usepackage{float}
\usepackage[style=ieee,backend=biber,defernumbers=true]{biblatex}

%% We encourage the use of mathptmx for consistent usage of times font
%% throughout the proceedings. However, if you encounter conflicts
%% with other math-related packages, you may want to disable it.

%% This turns references into clickable hyperlinks.
\usepackage[bookmarks,backref=true,linkcolor=black]{hyperref} %,colorlinks
\hypersetup{
  pdfauthor = {},
  pdftitle = {},
  pdfsubject = {},
  pdfkeywords = {},
  colorlinks=true,
  linkcolor= black,
  citecolor= black,
  pageanchor=true,
  urlcolor = black,
  plainpages = false,
  linktocpage
}

%% If you are submitting a paper to a conference for review with a double
%% blind reviewing process, please replace the value ``0'' below with your
%% OnlineID. Otherwise, you may safely leave it at ``0''.
\onlineid{0}

%% declare the category of your paper, only shown in review mode
\vgtccategory{Research}

%% allow for this line if you want the electronic option to work properly
\vgtcinsertpkg

%% In preprint mode you may define your own headline.
\preprinttext{}%To appear in an IEEE VGTC sponsored conference.}

%% Paper title.

\title{IEEE VIS 2016 Tutorial Proposal}

%% This is how authors are specified in the journal style

%% indicate IEEE Member or Student Member in form indicated below
\author{Jun Tao, Hanqi Guo, Bei Wang, Christoph Garth, and Tino Weinkauf}
\authorfooter{
%% insert punctuation at end of each item
\item
 Jun Tao is with University of Notre Dame. E-mail: jtao1@nd.edu.
\item
 Hanqi Guo is with Argonne National Laboratory. Email: hguo@anl.gov.
\item
 Bei Wang is with University of Utah. Email: beiwang@sci.utah.edu.
\item
 Christoph Garth is with University of Kaiserslautern. Email: garth@cs.uni-kl.de.
\item
 Tino Weinkauf is with KTH Stockholm. Email: weinkauf@kth.se.
}

%other entries to be set up for journal
% \shortauthortitle{Biv \MakeLowercase{\textit{et al.}}: Global Illumination for Fun and Profit}
%\shortauthortitle{Firstauthor \MakeLowercase{\textit{et al.}}: Paper Title}

%% Abstract section.
% \abstract{Duis autem vel eum iriure dolor in hendrerit in vulputate
% velit esse molestie consequat, vel illum dolore eu feugiat nulla
% facilisis at vero eros et accumsan et iusto odio dignissim qui blandit
% praesent luptatum zzril delenit augue duis dolore te feugait nulla
% facilisi. Lorem ipsum dolor sit amet, consectetuer adipiscing elit,
% sed diam nonummy nibh euismod tincidunt ut laoreet dolore magna
% aliquam erat volutpat. Ut wisi enim ad minim veniam, quis nostrud exerci tation ullamcorper
% suscipit lobortis nisl ut aliquip ex ea commodo consequat. Duis autem
% vel eum iriure dolor in hendrerit in vulputate velit esse molestie
% consequat, vel illum dolore eu feugiat nulla facilisis at vero eros et
% accumsan et iusto odio dignissim qui blandit praesent luptatum zzril
% delenit augue duis dolore te feugait nulla facilisi.
% } % end of abstract

%% Keywords that describe your work. Will show as 'Index Terms' in journal
%% please capitalize first letter and insert punctuation after last keyword
% \keywords{Radiosity, global illumination, constant time}

%% ACM Computing Classification System (CCS). 
%% See <http://www.acm.org/class/1998/> for details.
%% The ``\CCScat'' command takes four arguments.

% \CCScatlist{ % not used in journal version
%  \CCScat{K.6.1}{Management of Computing and Information Systems}%
% {Project and People Management}{Life Cycle};
%  \CCScat{K.7.m}{The Computing Profession}{Miscellaneous}{Ethics}
% }

% %% Uncomment below to include a teaser figure.
%   \teaser{
%  \centering
%  \includegraphics[width=16cm]{CypressView}
%   \caption{In the Clouds: Vancouver from Cypress Mountain.}
%   }

%% Uncomment below to disable the manuscript note
%\renewcommand{\manuscriptnotetxt}{}

%% Copyright space is enabled by default as required by guidelines.
%% It is disabled by the 'review' option or via the following command:
% \nocopyrightspace

%%%%%%%%%%%%%%%%%%%%%%%%%%%%%%%%%%%%%%%%%%%%%%%%%%%%%%%%%%%%%%%%
%%%%%%%%%%%%%%%%%%%%%% START OF THE PAPER %%%%%%%%%%%%%%%%%%%%%%
%%%%%%%%%%%%%%%%%%%%%%%%%%%%%%%%%%%%%%%%%%%%%%%%%%%%%%%%%%%%%%%%%

\newcommand{\addverticalspace}{\vspace{3mm}}

\bibliography{template}
\nocite{*}

\DeclareBibliographyCategory{Tao}
\addtocategory{Tao}{tao14fs,tao14fc,tao13uni,tao16fs,wang16fv}

\DeclareBibliographyCategory{Guo}
\addtocategory{Guo}{guo13tvcg,guo14tvcg,liu16pvis,guo16tvcg}

\DeclareBibliographyCategory{Wang}
\addtocategory{Wang}{skraba2016,skraba2015,skraba2014b,skraba2014a,wang2013}

\DeclareBibliographyCategory{Weinkauf}
\addtocategory{Weinkauf}{weinkauf07c,weinkauf10c,weinkauf12a,weinkauf11b,stoeter12}

\DeclareBibliographyCategory{Garth}
\addtocategory{Garth}{agranovsky2015,biedert2015,agranovsky2014,hummel2013,barakat2012}

\DeclareBibliographyCategory{Other}
\addtocategory{Other}{theisel03b}

% \DeclareFieldFormat{labelnumberwidth}{}
% \setlength{\biblabelsep}{0pt}


\begin{document}

%% The ``\maketitle'' command must be the first command after the
%% ``\begin{document}'' command. It prepares and prints the title block.

%% the only exception to this rule is the \firstsection command

\maketitle

\section*{TITLE}
Recent Advancements of Feature-based Flow Visualization and Analysis.

\section*{DURATION}
The tutorial is full-day, including 250 miniutes presentation, 50 minutes discussion, and 130 miniutes coffee break and lunch time.

\section*{SCHEDULE}

\vspace{-0.1in}
\begin{table}[H]
\begin{tabular}{lll}
Introduction & All & 10 minutes\\
Talk1 & Jun Tao & 50 minutes\\
Talk2 & Tino Weinkauf & 50 minutes\\
Break & -------- & 20 minutes\\
Talk3 & Bei Wang & 50 minutes\\
Lunch & -------- & 90 minutes\\
Talk4 & Christoph Garth & 50 minutes\\
Talk5 & Hanqi Guo & 50 minutes\\
Break & -------- & 20 minutes\\
Panel discussion & All & 50 minutes\\
Closing remarks & All & 10 minutes
\end{tabular}
\end{table}

\section*{ORGANIZER}

\vspace{-0.1in}
\begin{table}[H]
\begin{tabular}{ll}
Christoph Garth & University of Kaiserslautern\\
Hanqi Guo & Argonne National Laboratory\\
Jun Tao & University of Notre Dame\\
Bei Wang & University of Utah\\
Tino Weinkauf & KTH Stockholm
\end{tabular}
\end{table}

\section*{ABSTRACT}
Flow visualization has been a central topic in scientific visualization for many years, which can be explained by the ubiquity of vector fields in various kinds of scientific, engineering, medical researches. In all these domains, with today's ever-growing computation power, numerical simulations produce large, time-varying and highly complex vector fields. Preserving the rich information in these large and complex vector fields and presenting concise visualizations for clarity are two desired goals, but they are often conflicting. Striking a balance between them is challenging, which requires us to better distinguish the features from the contexts. Understanding and extracting features become critical to obtain insights from the vector fields with growing sizes and complexities.

In this tutorial, we cover different topics centered at the feature-based flow visualization and analysis: (a) interactive techniques that allow users to discover their features of interest; (b) spatio-temporal flow analysis that considers n-dimensional unsteady flows as (n+1)-dimensional steady flows; (c) feature extraction, tracking and simplification with robustness that captures structural stability of the data; (d) vector field techniques for large-scale time-varying data, especially the parallel algorithms and in-situ techniques; and (f) theories and scalability issues in ensemble and uncertain flow. This tutorial aims at providing information of the state-of-the-art techniques for feature-based flow visualization in different aspects, including interactive exploration, large-scale time-varying data, topological robustness and ensemble data.

\section*{LEVEL}
Intermediate/Advanced.

\section*{PREREQUISITE}
A general understanding of flow fields and flow visualization, including basic concepts and techniques, such as different kinds of field lines, critical points, and particle tracing, etc.

% \section*{AUDIENCE}
% The intended audience includes students, researchers and practitioners who are interested in the recent advances in flow visualization. More details to be added.

\section*{DESCRIPTION}
The description and outline of each topic are presented as the followings:

\addverticalspace

\noindent\textbf{\textit{Expressive Flow Field Exploration}}\\
\textbf{Jun Tao}
\paragraph{Abstract}
A major task of visualizing steady flow fields is to allow users perceive the flow patterns and locate features of interest. Traditional flow visualization approaches, such as seed placement and streamline selection, generate an appropriate set of streamlines to describe steady flow fields. However, only limited capabilities are provided to meet specific needs from different users, especially at the streamline segment level. In this talk, we will start from an automatic streamline and viewpoint selection framework, and then introduce three interactive exploration approaches. We demonstrate that the transition from the automatic approaches to the interactive ones provides more flexibility that allows users to specify, identify and observe their interested patterns/features in a more desired way.

\addverticalspace

\noindent\textbf{\textit{Spatio-temporal Flow Analysis}}\\
\textbf{Tino Weinkauf}
\paragraph{Abstract}
Understanding the processes in time-dependent flows is of crucial importance in many domains. Different methods exist for this purpose. This talk reviews methods that build on a spatio-temporal concept where the temporal dimension is treated on equal footing with the spatial dimensions. This means, an n-dimensional unsteady flow is analyzed as an (n+1)-dimensional steady flow. This has led to a number of powerful analysis and visualization methods in the last decade such as Feature Flow Fields \cite{theisel03b, weinkauf11b}, Swirling Motion Cores \cite{weinkauf07c}, Streak Lines as Tangent Curves \cite{weinkauf10c, weinkauf12a}, and more. The talk will cover the range from theoretical foundations to the applications on real data.

\addverticalspace

\noindent\textbf{\textit{Flow Analysis with Robustness}}\\
\textbf{Bei Wang}
\paragraph{Abstract}
This talk will review topological approaches for flow visualization. In particular, we will discuss a recent line of research that spans feature extraction, feature tracking, and feature simplification of vector fields based upon the topological notion of robustness that captures structural stability of the data. Robustness, a concept similar to persistent homology, quantifies the stability of critical points with respect to the minimum amount of perturbation in the fields required to remove them. We will discuss how this line of work can potentially increase the interpretability of data, specifically, by giving a coherent and multi-scale view of the flow dynamics under both stationary and time-varying settings. We will demonstrate how robustness-based approaches are independent of the topological skeleton and are scalable to large-scale datasets.

\addverticalspace

\noindent\textbf{\textit{Vector Field Techniques for Large-Scale Data}}\\
\textbf{Christoph Garth}
\paragraph{Abstract}
Large-scale vector fields as arising from modern scientific computing and experimental workflows pose substantial and significant challenges to visualization. While there is a rich body of work addressing flow visualization, many methods are unable to scale to modern data set sizes both algorithmically and with respect to the complexity of the obtained results. In his talk, he will discuss the application of flow visualization techniques to large-scale, time-varying vector fields, and report on recent research results in this area. Particular attention will be given to parallel algorithms and in-situ techniques that eschew the requirement to store full-fidelity data to achieve accurate visualization. On the latter topic, the talk will discuss several methods to flexibly analyse vector field data at reduced resolution. To conclude, recent results for large-scale vector field ensembles will be discussed.

\addverticalspace

\noindent\textbf{\textit{Theories and Scalability Issues in Ensemble and Uncertain Flow Visualization}}\\
\textbf{Hanqi Guo}
\paragraph{Abstract}
This talk will cover the topics on theories and scalability issues in ensemble and uncertain flow visualization. On one hand, we must redefine features in flow visualization, such as FTLE and LCS, for ensemble and uncertain flows. On the other hand, we have to scale up for the ensemble and uncertain flow analysis with supercomputers. We are going to review the literatures on this topic and present our recent studies.


\noindent\textbf{\textit{Scalable Ensemble and Uncertain Flow Field Visualization}}\\
\textbf{Hanqi Guo}
\paragraph{Abstract}
This talk covers both theoritical foundations and scalability studies in 
ensemble and uncertain flow visualization. 
As the growth of computation powers, scientists can generate ensembles of flow simulations, or flows with uncertainties, but it remains a great challenge to visualize and understand such data. 
First, features in flow visualization, such as FTLE and LCS, must be redefined for ensemble and uncertain flows.  
We review the traditional and direct visualization techniques, as well as advances in this topic, such as coupled field line tracing and analysis in numerical ensembles, comparative ensemble flow visualization, and the the measurement of flow divergence in uncertain unsteady flows. 
Second, the analysis of uncertain and ensemble and uncertain flows require scalability. 
We review the scalable algorithms to advect particles in ensemble and uncertain flows, 
because they play the central role in flow analysis and consumes majority computation time. 
New techniques in flow data management and stochastic particle tracing are covered, which can help scientists analyze ensemble and uncertain flows at scale. 


\section*{TUTORIAL NOTES}
The tutorial notes will consist of the description of the tutorial, copies of the slides for each talk, and an extensive bibliography including specific references used in the tutorial as well as a general selection of relevant references.

\section*{SPEAKERS}
The background of each speaker is listed in alphabetical order.

\addverticalspace

\noindent \textbf{Christoph Garth}\\
\emph{University of Kaiserslautern}\\
\href{mailto:garth@cs.uni-kl.de}{garth@cs.uni-kl.de}\\
\url{http://vis.uni-kl.de/people/garth/}

\addverticalspace

Christoph Garth is an assistant professor in the Computer Science Dept. at the University of Kaiserslautern, Germany. His main research interests include visualization and analysis of large-scale, multi-modal data as well as time-varying vector field visualization, with an emphasis on topology-based methods and in situ techniques.

\printbibliography[title={Relevant Publications},category=Garth]

\noindent \textbf{Hanqi Guo}\\
\emph{Argonne National Laboratory}\\
\href{mailto:hguo@anl.gov}{hguo@anl.gov}\\
\url{http://www.mcs.anl.gov/~hguo/}

\addverticalspace

Hanqi Guo is a Postdoctral Appointee in the Mathematics and Computer Science Division, Argonne National Laboratory. He received his PhD degree in computer science from Peking University in 2014, and the BS degree in mathematics and applied mathematics from Beijing University of Posts and Telecommunications in 2009. His research interests are mainly on uncertainty visualization, flow visualization, and large-scale scientific data visualization.

\printbibliography[title={Relevant Publications},category=Guo]

\noindent \textbf{Jun Tao}\\
\emph{University of Notre Dame}\\
\href{mailto:jtao1@nd.edu}{jtao1@nd.edu}\\
\url{http://www.nd.edu/~jtao1/}

\addverticalspace

Jun Tao is currently a postdoctoral researcher at University of Notre Dame. He received a PhD degree in computer science from Michigan Technological University in 2015. His major research interest is scientific visualization, especially on applying information theory, optimization techniques, and topological analysis to flow visualization and multivariate data exploration. He is also interested in graph-based visualization, image collection visualization, and software visualization. He received the Dean’s Award for Outstanding Scholarship and the Finishing Fellowship at Michigan Technological University in 2015, and a Best Paper Award at IS\&T/SPIE VDA 2013.

\printbibliography[title={Relevant Publications},category=Tao]

\noindent \textbf{Bei Wang}\\
\emph{University of Utah}\\
\href{mailto:beiwang@sci.utah.edu}{beiwang@sci.utah.edu}\\
\url{http://www.sci.utah.edu/~beiwang/}

\addverticalspace
Bei Wang is an assistant professor at the School of Computing and the Scientific Computing and Imaging Institute, University of Utah. Her main research interests lie in the theoretical, algorithmic, and application aspects of data analysis and data visualization, with a focus on topological techniques. She is also interested in computational biology and bioinformatics, machine learning and data mining. She is a member of ACM and IEEE.
\addverticalspace

\printbibliography[title={Relevant Publications},category=Wang]

\noindent \textbf{Tino Weinkauf}\\
\emph{KTH Stockholm}\\
\href{mailto:weinkauf@kth.se}{weinkauf@kth.se}\\
\url{http://www.csc.kth.se/~weinkauf/}

\addverticalspace
Tino Weinkauf received his diploma in computer science from the University of Rostock in 2000. From 2001, he worked on feature-based flow visualization and topological data analysis at Zuse Institute Berlin. He received his Ph.D. in computer science from the University of Magdeburg in 2008. In 2009 and 2010, he worked as a postdoc and adjunct assistant professor at the Courant Institute of Mathematical Sciences at New York University. He started his own group in 2011 on Feature-Based Data Analysis in the Max Planck Center for Visual Computing and Communication, Saarbrücken. Since 2015, he holds the Chair of Visualization at KTH Stockholm. His current research interests focus on flow analysis, discrete topological methods, and information visualization.

\printbibliography[title={Relevant Publications},category=Weinkauf]

\printbibliography[title={Other References},category=Other]

% %% if specified like this the section will be committed in review mode
% \acknowledgments{
% The authors wish to thank A, B, C. This work was supported in part by
% a grant from XYZ.}

% \bibliographystyle{abbrv}
% %%use following if all content of bibtex file should be shown
% %\nocite{*}
% \bibliography{template}
\end{document}
